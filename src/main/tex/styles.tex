
\documentclass[a4paper, titlepage]{article}
\usepackage[left=2cm,right=2cm,
    top=2cm,bottom=2cm,bindingoffset=0cm]{geometry}
    
\usepackage{pscyr}
\DeclareSymbolFont{T2Aletters}{\encodingdefault}{\rmdefault}{m}{it}
\usepackage[warn]{mathtext}          % русские буквы в формулах, с предупреждением
\usepackage[T2A]{fontenc}            % внутренняя кодировка  TeX
\usepackage[utf8x]{inputenc}         % кодовая страница документа
\usepackage[english, russian]{babel} % локализация и переносы
\usepackage{indentfirst}   % русский стиль: отступ первого абзаца раздела
\usepackage{misccorr}      % точка в номерах заголовков
\usepackage{cmap}          % русский поиск в pdf
\usepackage{color}
\usepackage{graphicx}      % Работа с графикой \includegraphics{}
\usepackage{psfrag}        % Замена тагов на eps картинкаx
\usepackage{caption2}      % Работа с подписями для фигур, таблиц и пр.
\usepackage{color, soul}   % Разряженный текст \so{} и подчеркивание \ul{}
\usepackage{soulutf8}      % Поддержка UTF8 в soul
\usepackage{fancyhdr}      % Для работы с колонтитулами
\usepackage{multirow}      % Аналог multicolumn для строк
\usepackage{ltxtable}      % Микс tabularx и longtable
\usepackage{paralist}      % Списки с отступом только в первой строчке
\usepackage{longtable}
\usepackage{tabularx}
\usepackage{listings}      % листинги
\usepackage[perpage]{footmisc} % Нумерация сносок на каждой странице с 1

\usepackage{hyperref}
\hypersetup{colorlinks,urlcolor=blue}

\usepackage{amsmath}
\usepackage{amsfonts}
\usepackage{amssymb}

\usepackage{float}
\graphicspath{{images/}}
\usepackage{multirow}
\usepackage{hyperref}
\usepackage{cite}
\usepackage{verbatim} % блоки комментариев \begin{comment}
\usepackage{icomma}  % В России для разделения целой части цисла от дробной используется запятая '','', в англоязычных странах используется точка ''.''. Если следовать русской традиции набора чисел, используя в качестве разделителя запятую, то получается большой пробел меджу запятой и дробной частью, т.к. LaTeX добавляет небольшой пробел после запятой. Пакет icomma позволяет убрать этот эффект

\usepackage{listings}

\newcommand*\dd{\mathop{}\!\mathrm{d}}
\newcommand*\Dd[1]{\mathop{}\!\mathrm{d^#1}}
\newcommand*\Cyr[1]{\text{\textit{#1}}}


\definecolor{dkgreen}{rgb}{0,0.6,0}
\definecolor{gray}{rgb}{0.5,0.5,0.5}
\definecolor{mauve}{rgb}{0.58,0,0.82}


\renewcommand{\lstlistingname}{Листинг}

\newcommand{\Code}{\fontsize{12}{15}\selectfont}

\lstset{ 
  language=Java,                % Язык программирования 
  %numbers=left,
  numberstyle=\color{gray},
  frame=single,
  title=\lstname,
  rulecolor=\color{black},  
  keywordstyle=\color{blue},          % Стиль ключевых слов
  commentstyle=\color{dkgreen},       % Стиль комментариев
  stringstyle=\color{mauve},
  extendedchars=\true,
  inputencoding=utf8,
  basicstyle=\Code\ttfamily,
  showtabs=\true,
  showspaces=\true,
  columns=fixed,
  breaklines=true,                        % Automatic line breaking?
  breakatwhitespace=true,                % Automatic breaks only at whitespace?
  keepspaces=true,
  extendedchars=\true,
  inputencoding=utf8x,
  literate={а}{{\selectfont\char224}}1
{б}{{\selectfont\char225}}1
{в}{{\selectfont\char226}}1
{г}{{\selectfont\char227}}1
{д}{{\selectfont\char228}}1
{е}{{\selectfont\char229}}1
{ё}{{\"e}}1
{ж}{{\selectfont\char230}}1
{з}{{\selectfont\char231}}1
{и}{{\selectfont\char232}}1
{й}{{\selectfont\char233}}1
{к}{{\selectfont\char234}}1
{л}{{\selectfont\char235}}1
{м}{{\selectfont\char236}}1
{н}{{\selectfont\char237}}1
{о}{{\selectfont\char238}}1
{п}{{\selectfont\char239}}1
{р}{{\selectfont\char240}}1
{с}{{\selectfont\char241}}1
{т}{{\selectfont\char242}}1
{у}{{\selectfont\char243}}1
{ф}{{\selectfont\char244}}1
{х}{{\selectfont\char245}}1
{ц}{{\selectfont\char246}}1
{ч}{{\selectfont\char247}}1
{ш}{{\selectfont\char248}}1
{щ}{{\selectfont\char249}}1
{ъ}{{\selectfont\char250}}1
{ы}{{\selectfont\char251}}1
{ь}{{\selectfont\char252}}1
{э}{{\selectfont\char253}}1
{ю}{{\selectfont\char254}}1
{я}{{\selectfont\char255}}1
{А}{{\selectfont\char192}}1
{Б}{{\selectfont\char193}}1
{В}{{\selectfont\char194}}1
{Г}{{\selectfont\char195}}1
{Д}{{\selectfont\char196}}1
{Е}{{\selectfont\char197}}1
{Ё}{{\"E}}1
{Ж}{{\selectfont\char198}}1
{З}{{\selectfont\char199}}1
{И}{{\selectfont\char200}}1
{Й}{{\selectfont\char201}}1
{К}{{\selectfont\char202}}1
{Л}{{\selectfont\char203}}1
{М}{{\selectfont\char204}}1
{Н}{{\selectfont\char205}}1
{О}{{\selectfont\char206}}1
{П}{{\selectfont\char207}}1
{Р}{{\selectfont\char208}}1
{С}{{\selectfont\char209}}1
{Т}{{\selectfont\char210}}1
{У}{{\selectfont\char211}}1
{Ф}{{\selectfont\char212}}1
{Х}{{\selectfont\char213}}1
{Ц}{{\selectfont\char214}}1
{Ч}{{\selectfont\char215}}1
{Ш}{{\selectfont\char216}}1
{Щ}{{\selectfont\char217}}1
{Ъ}{{\selectfont\char218}}1
{Ы}{{\selectfont\char219}}1
{Ь}{{\selectfont\char220}}1
{Э}{{\selectfont\char221}}1
{Ю}{{\selectfont\char222}}1
{Я}{{\selectfont\char223}}1
}
\fontsize{14}{16pt}\selectfont
