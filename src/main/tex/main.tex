\include{styles}
\title{Моделирование движения двух тел в поле силы тяжести}
\author{Крамарев А.Г.}

\begin{document} % Маркер начала документа
\maketitle % Создать титульный лист на основе данных в заголовке документа

\section{Задача}

Взаимодействие двух частиц
\begin{equation}
\label{eq:main}
\begin{cases}
m_1 \ddot{x}_1 = G \dfrac{ m_1 m_2 (x_2 - x_1) }{r^3} \\ 
m_1 \ddot{y}_1 = G \dfrac{ m_1 m_2 (y_2 - y_1) }{r^3} \\ 
\dot{x}_1 = v_1 \\
\dot{y}_1 = v_1
\end{cases}
\end{equation}
где $r = \sqrt{(x_1 - x_2)^2 + (y_1 - y_2)^2}$ --- расстояние между частицами для двухмерного случая \\
$m_1, m_2$ --- массы первого и второго тел \\

Решим задачу с помощью явного метода Эйлера:

\begin{equation}
\begin{cases}

    \dfrac{x^{n+1}-x^n}{\tau} = V_{1x}^n(t) \\[1em]
    \dfrac{y^{n+1}-y^n}{\tau} = V_{1y}^n(t) \\[1em]
    
    \dfrac{V_{1x}^{n+1}(t) - V_{1x}^n(t)}{\tau} = -G\dfrac{m_2(y_2^n - y_1^n)}{r^3} \\[1em]
    \dfrac{V_{1y}^{n+1}(t) - V_{1y}^n(t)}{\tau} = -G\dfrac{m_2(y_2^n - y_1^n)}{r^3} \\

\end{cases}
\end{equation}



\end{document} % Маркер завершения документа
