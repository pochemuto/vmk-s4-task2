\include{styles}
\title{Движение планет}
\author{Крамарев А.Г.}

\begin{document} % Маркер начала документа
\maketitle % Создать титульный лист на основе данных в заголовке документа

\section{Задача}

Взаимодействие двух частиц
\begin{equation}
\label{eq:main}
\begin{cases}
m_1 \ddot{x}_1 = G \dfrac{ m_1 m_2 (x_2 - x_1) }{r^3} \\ 
\dot{x}_1 = v_1
\end{cases}
\end{equation}
где $r = \sqrt{(x_1 - x_2)^2 + (y_1 - y_2)^2}$ --- расстояние между частицами для двухмерного случая 

\section{Обезразмеривание}
\subsection{Масса}
Характерной массой частицы примем массу первой частицы $m_1$:
$$
    m_\Cyr{хар} = m_1
$$ $$
    \hat{m} = m \cdot m_1
$$

\subsection{Расстояние}
За характерное расстояние будет брать расстояние между точками:
$$
    x_\Cyr{хар} = r
$$ $$
    \hat{x} = x \cdot r
$$
\subsection{Гравитационная постоянная}
$$
 G_\Cyr{хар} = G
$$

Обрезразмерим уравнение \ref{eq:main}:
$$
    \begin{cases}
        \ddot{x}_1 = -G
    \end{cases}
$$


\end{document} % Маркер завершения документа
